\noindent
{
	\color{red}
The concluding chapter of a dissertation is often underutilised because it
is too often left too close to the deadline: it is important to allocation
enough attention.  Ideally, the chapter will consist of three parts:

\begin{enumerate}
\item (Re)summarise the main contributions and achievements, in essence
      summing up the content.
\item Clearly state the current project status (e.g., ``X is working, Y
      is not'') and evaluate what has been achieved with respect to the
      initial aims and objectives (e.g., ``I completed aim X outlined
      previously, the evidence for this is within Chapter Y'').  There
      is no problem including aims which were not completed, but it is
      important to evaluate and/or justify why this is the case.
\item Outline any open problems or future plans.  Rather than treat this
      only as an exercise in what you {\em could} have done given more
      time, try to focus on any unexplored options or interesting outcomes
      (e.g., ``my experiment for X gave counter-intuitive results, this
      could be because Y and would form an interesting area for further
      study'' or ``users found feature Z of my software difficult to use,
      which is obvious in hindsight but not during at design stage; to
      resolve this, I could clearly apply the technique of Smith [7]'').
\end{enumerate}
}

\section{Summary of Achievements}
My intention of recreating the genetic algorithm from Thompson's work on
evolutionary hardware \cite{10.1007/3-540-63173-9_61} and applying it to the
binary arithmatic problem was successful, as outlined in Chapter~\ref{chap:evaluation}.
The inital genetic algorithm parameters in Section~\ref{s:ga_tune} mirror
Thompson's exactly. When applied to 2-bit binary addition the algorithm
executed quickly but in 30 repeated executions to generation 15000, failed to produce a perfect
binary ADDer. The average final accuracy was 40, out of a maximum of 48.

This baseline algorithm was then improved on by factoring in features from
modern genetic algorithm and evolvable hardware literiture.
These features included; a
multi-objective fitness function incorperating a diversity measurement,
mutation rate variation, tournament selection, varied crossover probability,
the absence of elitism, population size, multi-objective weightings, and
coevolution (with varied virulence). Assessing each of these in turn
a 2-bit binary addition specific genetic algorithm was devised with vastly
improved performance. In repeated experiments almost 1/4 of executions to
generation 15000 produced a perfect solution will 100\% accuracy. This
executed in comparable time to the initial parameters specified in
\cite{10.1007/3-540-63173-9_61}, and achieved an average final fitness of 44,
a clear improvement.

\todo directly compare to the paper where tournament selection was used to evolve an adder

The search space of FPGAs defined by 32 byte strings is huge. Clearly the
scheme devised in this dissertation is a vast improvment over innumerated bruteforce
search. \todo stats analysis of number of runs required to be 99\% sure of producing a
result.

This system was then exposed to a series of dynamic problems; simple fault
recovery, ``sticky" fault mitigation, and shifting weighting optimisation.
The system coped admirably with simple fault recovery; when a critical fault
was injected in the FPGA simulation the accuracy dropped to a cripplingly low.
In repeated executions, by 14000 generations evolving with the fault 43\%
had developed perfect solutions, and the average accuracy was 42. The success
rate here is incredibly high when compared to the performance of the underlying
evolvable hardware system which achieved perfect answers on 23\% of repeated executions.
With a single oscilating ``sticky" fault on 7\% the genetic algorithm was
able to divise a solution which worked both in the presence of the fault and without
it. Even with two faults clear improvments in how the system coped with shifting
correctness critera are visible.

The entire evolvable platform is available on GitHub. The FPGA simulation
is a distinct component which exposes concise functionality to the evolutionary
software component. The simulation can be downloaded independently and used
not only as a component in other evolutionary systems but in any software project
which could make use of a streamlined generic FPGA simulation of arbitrary developer
defined size. The usefulness of the ability to convert binary strings into FPGAs and evaluate specified
configurations extends beyond the context presented here and could form the backbone
of a plethora of speculative hardware projects. This fits the FPGA simulation
specification outlined in the project aims.

Exploration into the scaling problems of evolvable hardware outlined the key
offenders contributing to extreme execution times. With simulated hardware
the execution time scaled nicely with FPGA size, but the clear culprit in
slow evaluation cycles was the sheer number of trails conducted for larger
problems. Coevolution was proposed as a mitigation strategy, and would reduce
the number of trials conducted; however, performance under the 2-bit problem
was severly dissapointing and was completely unsuitable as a replacement to
conventional exhaustive testing. In order to achieve any sort of success
elitism was non-optional and this required uniform exhaustive testing to function
properly. Even without the extra computation required for elitism, the overheads
maintaining a second distinct evolutionary population lead to extreme average
execution times.

A GUI was developed to provide visual feedback on the population health
and genetic algorithm performance in an attempt to improve user understanding
of the underlying process and drive parameter choices. This was somewhat successful,
but the vast quantity of information on hand had to be reduced; the only detailed
information on offer was for the current generation's best performing individual.
For this an exhaustive list of test performance and a diagram of the FPGA configuration
was provided to the user. The remaining population qualities were reduced to a handful
of statistics; generation number, mean accuracy, and mean diversity. More innovation
is required in this domain to truly offer a way to monitor a population and guide
parameter choice.

There are a great number of limitations existing within this project, and
evolvable hardware in a wider sense. 2-bit addition to most would seem trivial,
and even though the execution time was painlessly short, directly extending
this to 3-bit addition required 1473s and 4-bit addition 8537s. Designing an
itterative design system around any larger problem than 2-bit binary arithmatic
is simply not an option. Extending
any assumptions here to useful applications requires solving a great number of
challenges.

Visual inspection of many of the accuracy graphs in Chapter~\ref{chap:evaluation}
could lead a reader to believe that many improvments are simply lucky leaps from
a general background population which usualy performs moderatly; and the reader
would not be wrong. Population quality tends to reach a certain value and then
remain consistent, and best-case improvements become fewer and far between as
execution continues. The only way a system is saved from backtacking is by the
inclusing of an elitism mechanic. These together indicate that the population improves
until a certain value and then maintains a primordeal soup, rarely performing well
but on occasion generating an example with an improved performance. The cycle repeats
but making improvements requires larger and larger leaps from this primordeal soup.

The disparity between vanilla evolutionary hardware in Figure~\ref{fig:div_2}, only
achieving a success rate of 23\% and the fault injection examples in
Figure~\ref{fig:simple_fault} achieving 43\% highlights the tendancy for
this evolvable hardware incarnation to stumble blindly into evolutionary deadends,
often dwelling needlessly on bloated hopeless solutions. These solutions take up
too many of the resources alloted to the FPGA to achieve their mostly-good results
to progress to perfection.

The system optimisation under dynamicly shifting weightings is underwhelming,
until the system is placed under real pressure the evolutionary tactic seems to be
to maintain the current solution rather than perform minute tweaks to gain a performance
advantage.

\todo mention the domain specific knowledge required for hardware design is
replaced by domain specific knowledge for GA design.

\section{Project state}
A fully flexible binary arithmatic evolutionary hardware platform works, with
a plethora of options and configurable components. The FPGA simulator and GUI are a distinct
self contained entity allowing for use in machine learning hardware design beyond
genetic algorithms.

\section{Further work}
Given the self contained nature of the underlying FPGA simulation there is scope to
extend this project with more machine learning mechanism to optimise the FPGA
configuration for the binary addition problem. Further study into the underlying
structure of the search space could provide insights which would allow further
tuning of the genetic algorithm or, more likely, open the door to more esoteric
specific implimentations of other techniques, such as neural networks.

Remaning close to genetic algorithms the nature of the mutation function
could shift. Rather than comprising random bitflips there could be a chance
of a predefined operation with knowledge of the phenotype to genotype mapping,
such as; cell translation/rotation, cell reording, or mirroring cell connections.
Such a modification would move this exploration to the fringes of genetic algorithm
research, if not beyond the realm of, as genetic algorithms conventionally do not
have any in-build understanding of what the genetic material represents (beyond the
evaluation step).

If the plethora of issues surrounding system scaling could be addressed
one could concieve of a hardware system similar to what was proposed in
\cite{10.1007/3-540-61093-6_6}, save for a general purpose processor.
Specific fault-prone units could be replaced with micro-FPGAs, each preloaded
with the original component specification. Upon the detection of a fault
the genetic algorithm begins. This itterates over the design finding ways to
capitalise on the demonstrated fault recovery performance and devise a unit
which works perfectly in the presence of the fault. The vast majority of modern systems are multi-core
so it is not unthinkable that should a component in one core fail any alternative
cores could initiate and conduct the self-healing process. These systems
could take advantage of evolvable hardware accelerator chips to create
truly rugged circuitry. The evolutionary search could be conducted in
CPU idling time, and even if it takes hours, a component which wouldn't
otherwise work could be opperational; ideal for satalites and/or operations
working on a scale such that brief component downtime for the sake of
longevity is not a huge issue.

If improvements in algorithm design leads to an evolvable hardware system
which, when confronted by a dynamically shifting weighted problem, can
perform prompt ajustments (at a speed faster than presented in this document)
then a flexible pseudo-FPGA-CPU as outlined above could facilitate
user/program specific fine tuning. Given contextual information (which shifts
the desirablility of certain functions and therefore the weightings) the
processor could recalibrate to better suit it's function. Suppose,
as a trivial example, user A performs a great deal of multiplication,
and user B loves subtraction; given this information the processor should
be able to optimise for each. Rather than trying to be accurate the processor
could be assumed to be constantly accurate (other wise it will not change)
and is instead trying to optimise for execution time.
performs a lot of multiplication
