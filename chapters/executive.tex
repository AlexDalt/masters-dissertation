The intersection between machine learning and hardware design is an oft-unexplored
area. Evolutionary hardware is one such field which strives
to automate hardware design through the application of genetic algorithms.
This thesis
explores how genetic algorithms can be improved in the context
of binary arithmetic.
The resulting evolutionary hardware systems are subjected to a series of
dynamic problems, including fault injection and contextual hardware optimisation.

Evolutionary hardware systems are built on Field Programmable Gate Arrays (FPGAs),
a configurable flexible hardware platform.
To improve development
time, and remove execution bottlenecks, a simulated FPGA will be constructed.
This will act as a self-contained unit and provide the evaluation back end to the genetic algorithm.

The research hypothesis is that genetic algorithms can be used to construct
efficient hardware systems suited for tackling the suite of dynamic problems
hardware encounters on a regular basis.

The main achievements include:
\begin{itemize}
	\item Building a highly specialised FPGA simulator (in C) to allow quick
		genetic algorithm development. FPGA size is arbitrary and subject to
		user specification.
	\item Implementing and improving on known genetic algorithms to
		create FPGA configurations for the binary arithmatic problem. The system is
		highly configurable with multi-objective training weightings, selection schemes,
		population size, mutation rate, elitism, coevolution (parasite size, and virulence),
		and crossover probability all subject to the whims of the user.
	\item Exploring fault recovery and dynamic optimisation with evolutionary
		hardware on the FPGA simulator.
	\item Creation of a clean user interface to allow for easy mid-execution user
		evaluation.
	\item Explore scaling bottlenecks and potential for mitigation with coevolutionary
		techniques.
\end{itemize}
