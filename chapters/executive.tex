\noindent
{
	\color{red}
This section should pr\'{e}cis the project context, aims and objectives,
and main contributions (e.g., deliverables) and achievements; the same
section may be called an abstract elsewhere.  The goal is to ensure the
reader is clear about what the topic is, what you have done within this
topic, {\em and} what your view of the outcome is.

The former aspects should be guided by your specification: essentially
this section is a (very) short version of what is typically the first
chapter.  Note that for research-type projects, this {\bf must} include
a clear research hypothesis.  This will obviously differ significantly
for each project, but an example might be as follows:

\begin{quote}
My research hypothesis is that a suitable genetic algorithm will yield
more accurate results (when applied to the standard ACME data set) than
the algorithm proposed by Jones and Smith, while also executing in less
time.
\end{quote}

\noindent
The latter aspects should (ideally) be presented as a concise, factual
bullet point list.  Again the points will differ for each project, but
an might be as follows:

\begin{quote}
\noindent
\begin{itemize}
\item I spent $120$ hours collecting material on and learning about the
      Java garbage-collection sub-system.
\item I wrote a total of $5000$ lines of source code, comprising a Linux
      device driver for a robot (in C) and a GUI (in Java) that is
      used to control it.
\item I designed a new algorithm for computing the non-linear mapping
      from A-space to B-space using a genetic algorithm, see page $17$.
\item I implemented a version of the algorithm proposed by Jones and
      Smith in [6], see page $12$, corrected a mistake in it, and
      compared the results with several alternatives.
\end{itemize}
\end{quote}
}


The intersection between machine learning and hardware design is an oft-unexplored
area. Evolutionary hardware is one such field which strives
to automate hardware design through the application of genetic algorithms.
This thesis
explores how genetic algorithms can be improved in the context
of binary arithmatic.
The resulting evolutionary hardware systems are subjected to a series of
dynamic problems, including fault injection and contextual hardware optimisation.

Evolutionary hardware systems are built on Field Programmable Gate Arrays (FPGAs),
a configurable flexible hardware platform.
To improve development
time, and remove execution bottlenecks, a simulated FPGA will be constructed.
This will act as a self-contained unit and provide the evaluation back end to the genetic algorithm.

My research hypothesis is that genetic algorithms can be used to construct
efficient hardware systems suited for tackling the suite of dynamic problems
hardware encounters on a regular basis.

The main achievements include:
\begin{itemize}
	\item Building a highly specialised FPGA simulator (in C) to allow quick
		genetic algorithm development. FPGA size is arbitrary and subject to
		user specification.
	\item Implementing and improving on known genetic algorithms to
		create FPGA configurations for the binary arithmatic problem. The system is
		highly configurable with multi-objective training weightings, selection schemes,
		population size, mutation rate, elitism, coevolution (parasite size, and virulence),
		and crossover probability all subject to the whims of the user.
	\item Exploring fault recovery and dynamic optimisation with evolutionary
		hardware on the FPGA simulator.
	\item Creation of a clean user interface to allow for easy mid-execution user
		evaluation.
	\item Explore scaling bottlenecks and potential for mitigation with coevolutionary
		techniques.
\end{itemize}
